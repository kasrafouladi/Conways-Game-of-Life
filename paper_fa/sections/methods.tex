\section{روش ها}

\subsection{توصیف و پیش‌پردازش داده‌ها}

در این پژوهش از پایگاه داده GISTEMP v4 متعلق به موسسه مطالعات فضایی گودارد ناسا استفاده شده است \cite{gistemp}. این پایگاه داده شامل بی‌هنجاری دمای میانگین جهانی به صورت ماهانه از سال 1880 تا امروز است که نسبت به دوره پایه 1951 تا 1980 محاسبه شده است. برای این مطالعه، داده‌های سالانه بی‌هنجاری دما در بازه 1950 تا 2020 استخراج شده است.

پیش از تحلیل، داده‌ها پاک‌سازی شده و مقادیر گمشده (NaN) حذف گردیده‌اند. میانگین جهانی با میانگین‌گیری روی تمام نقاط شبکه محاسبه شده است. برای مقایسه منصفانه روش‌ها، نقاط نمونه با فواصل یکنواخت و با استفاده از درون‌یابی خطی تولید شده‌اند تا همه روش‌ها روی داده‌های ورودی یکسان اجرا شوند.

\subsection{انتخاب بازه‌های آموزشی}

برای ارزیابی پایداری و تعمیم‌پذیری هر روش، آزمایش‌ها روی دو بازه آموزشی متفاوت انجام شده است: [1950, 2020] و [1960, 2010]. بازه اول کل دوره مورد نظر را پوشش می‌دهد و بازه دوم بر بخش مرکزی داده‌ها تمرکز دارد. این رویکرد امکان بررسی عملکرد روش‌ها در درون‌یابی (درون بازه آموزش) و برون‌یابی (خارج از بازه آموزش) را فراهم می‌کند.

\subsection{درون‌یابی محلی و نقطه‌ای}

برای هر سال هدف، درون‌یابی با استفاده از پنجره‌ای شامل ۸ همسایه نزدیک انجام می‌شود. این رویکرد مزایای زیر را دارد:
\begin{itemize}
    \item \textbf{افزایش دقت:} با تمرکز بر داده‌های نزدیک، مدل بهتر روند محلی را دنبال می‌کند و اثر نقاط دورتر کاهش می‌یابد.
    \item \textbf{کاهش نوسانات:} چندجمله‌ای‌های محلی کمتر دچار پدیده رانگه و ناپایداری عددی می‌شوند.
    \item \textbf{سازگاری با تغییرات محلی:} مدل می‌تواند به تغییرات محلی داده‌ها واکنش نشان دهد که در داده‌های واقعی و نویزی اهمیت دارد.
\end{itemize}
در صورت کمبود همسایه، اندازه پنجره یا درجه چندجمله‌ای به طور خودکار کاهش می‌یابد تا پایداری عددی حفظ شود.

\subsection{تکنیک های درون‌یابی}

روش‌های زیر پیاده‌سازی و مقایسه شده‌اند:
\begin{itemize}
    \item \textbf{نیوتن پیشرو محلی:} ساخت چندجمله‌ای با استفاده از تفاضلات پیشرو، مناسب نقاط ابتدای بازه \cite{atkinson1989introduction, burden2011numerical}.
    \item \textbf{نیوتن پسرو محلی:} مشابه روش پیشرو اما با تفاضلات پسرو، مناسب نقاط انتهایی بازه.
    \item \textbf{لاگرانژ محلی:} ساخت چندجمله‌ای عبوری از پنجره محلی، با دقت بالا در نواحی متراکم داده \cite{atkinson1989introduction}.
    \item \textbf{رگرسیون چندجمله‌ای محلی:} برازش چندجمله‌ای با روش کمترین مربعات روی پنجره محلی و نرمال‌سازی داده‌ها برای پایداری عددی \cite{brown2021polynomial}.
\end{itemize}

همه روش‌ها با استفاده از کتابخانه‌های numpy، scipy و matplotlib در پایتون پیاده‌سازی شده‌اند. در صورت شکست برازش چندجمله‌ای، مدل به طور خودکار به درون‌یابی خطی بازمی‌گردد.

\subsection{ارزیابی و خروجی}

برای هر روش و بازه، مقادیر پیش‌بینی‌شده با داده‌های واقعی مقایسه و نمودارها رسم و ذخیره شده‌اند. عملکرد عددی با معیار ریشه میانگین مربعات خطا (RMSE) سنجیده شده است.
