\section{Conclusion}

In this study, we compared several classical numerical interpolation techniques for reconstructing global mean temperature anomaly data. Our results demonstrate that while all methods can approximate the overall trend, local polynomial regression provides superior accuracy and robustness for climate data analysis. The use of local, pointwise interpolation windows for each prediction year significantly improves accuracy and stability, especially in the presence of noise and nonstationary trends. The GISTEMP dataset \cite{gistemp} serves as a reliable benchmark for such studies.

By evaluating methods across different training intervals, we highlighted the importance of interval selection and the sensitivity of interpolation methods to boundary effects. Additional steps such as data cleaning, normalization, and robust error handling further contribute to the reliability of the analysis.

Future work may explore the application of advanced machine learning-based interpolation methods and the extension of these techniques to spatially resolved climate datasets.