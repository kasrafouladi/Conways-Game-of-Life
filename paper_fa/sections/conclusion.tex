\section{جمع‌بندی}

در این پژوهش، چندین روش کلاسیک درون‌یابی عددی برای بازسازی داده‌های بی‌هنجاری دمای میانگین جهانی مقایسه شد. نتایج نشان داد که اگرچه همه روش‌ها قادر به تقریب روند کلی هستند، رگرسیون چندجمله‌ای محلی از نظر دقت و پایداری برای تحلیل داده‌های اقلیمی برتری دارد. استفاده از پنجره‌های محلی و نقطه‌ای برای هر سال پیش‌بینی، دقت و پایداری را به ویژه در حضور نویز و روندهای غیرایستا به طور قابل توجهی افزایش می‌دهد. پایگاه داده GISTEMP \cite{gistemp} به عنوان مرجع قابل اعتماد برای این مطالعات مورد استفاده قرار گرفت.

با ارزیابی روش‌ها در بازه‌های آموزشی مختلف، اهمیت انتخاب بازه و حساسیت روش‌ها به اثرات مرزی برجسته شد. همچنین، گام‌هایی مانند پاک‌سازی داده، نرمال‌سازی و مدیریت خطا به افزایش قابلیت اعتماد تحلیل کمک کرد.

پیشنهاد می‌شود در مطالعات آینده، از روش‌های پیشرفته مبتنی بر یادگیری ماشین و تعمیم این تکنیک‌ها به داده‌های مکانی-زمانی نیز استفاده شود.
