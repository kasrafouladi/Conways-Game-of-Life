\section{مقدمه}

تحلیل دقیق داده‌های اقلیمی برای درک پدیده گرمایش جهانی و پیامدهای آن اهمیت بالایی دارد. یکی از پرکاربردترین شاخص‌ها در علوم اقلیم، «بی‌هنجاری دما» است که بیانگر انحراف دمای مشاهده‌شده از یک مقدار مرجع یا دوره پایه می‌باشد \cite{gistemp, hansen2010global}. این رویکرد به پژوهشگران اجازه می‌دهد تغییرات دما را در مناطق و بازه‌های زمانی مختلف، مستقل از شرایط اقلیمی محلی، مقایسه کنند.

در این پروژه، تمرکز ما بر درون‌یابی داده‌های بی‌هنجاری دمای میانگین جهانی است که گامی کلیدی در بازسازی روندهای پیوسته اقلیمی از مشاهدات گسسته به شمار می‌رود. تکنیک‌های درون‌یابی برای پر کردن خلأهای داده، کاهش نویز و فراهم‌سازی امکان تحلیل‌های آماری بیشتر ضروری هستند \cite{atkinson1989introduction, burden2011numerical}. در این پژوهش، چندین روش کلاسیک درون‌یابی عددی شامل نیوتن پیشرو و پسرو محلی، لاگرانژ محلی و رگرسیون چندجمله‌ای محلی را مقایسه می‌کنیم و هر روش را از نظر دقت و کارایی در تحلیل داده‌های اقلیمی ارزیابی می‌نماییم.

ویژگی متمایز این کار، استفاده از درون‌یابی محلی و نقطه‌ای است؛ به این صورت که برای هر سال پیش‌بینی، درون‌یابی با استفاده از پنجره‌ای از نزدیک‌ترین همسایه‌ها انجام می‌شود و مدل سراسری ساخته نمی‌شود. این راهبرد محلی با هدف افزایش دقت، کاهش خطر بیش‌برازش یا نوسانات (مانند پدیده رانگه) و سازگاری با تغییرات محلی داده‌ها اتخاذ شده است. همچنین، با مقایسه روش‌ها در بازه‌های آموزشی مختلف مانند [1950, 2020] و [1960, 2010]، پایداری و تعمیم‌پذیری آن‌ها را در طول زمان بررسی می‌کنیم.

در ادامه، بخش دوم به معرفی داده‌ها و روش‌های درون‌یابی اختصاص دارد. در بخش سوم نتایج پیاده‌سازی روش‌ها ارائه می‌شود. بخش چهارم به بحث و مقایسه عملکرد روش‌ها می‌پردازد و در نهایت، بخش پنجم جمع‌بندی و پیشنهادها را بیان می‌کند.
