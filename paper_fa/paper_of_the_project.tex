\documentclass{article}
\usepackage{graphicx}
\usepackage{cite}
\usepackage{caption}
\usepackage{subcaption}
\usepackage{amsmath}
\usepackage{amsfonts}
\usepackage{hyperref}
\usepackage[xunicode,englishdigits]{xepersian}

\settextfont{IRNazanin}

\title{پروژه نهایی آنالیز عددی}
\author{کسری فولادی \\ ۴۰۲۴۴۲۲۹۷}
\date{\today}

\begin{document}

\maketitle

\small

\begin{abstract}
این مقاله به مقایسه جامع چندین روش کلاسیک درون‌یابی عددی در تحلیل داده‌های بی‌هنجاری دمای میانگین جهانی می‌پردازد. مفهوم بی‌هنجاری دما معرفی شده و مسئله در زمینه تحلیل داده‌های اقلیمی صورت‌بندی می‌شود. روش‌های نیوتن پیشرو و پسرو محلی، لاگرانژ محلی و رگرسیون چندجمله‌ای محلی با استفاده از داده‌های واقعی اقلیمی پیاده‌سازی و ارزیابی شده‌اند. دقت و ویژگی‌های هر روش مورد بحث قرار گرفته و بینش‌هایی درباره مناسب بودن آن‌ها برای تحلیل داده‌های علمی ارائه می‌شود.
\end{abstract}

\section{Introduction}

Accurate analysis of climate data is crucial for understanding global warming and its impacts. One of the most widely used metrics in climate science is the \textit{temperature anomaly}, which represents the deviation of the observed temperature from a reference value or baseline period \cite{gistemp, hansen2010global}. This approach allows researchers to compare temperature changes across different regions and time periods, independent of local climate variability.

In this project, we focus on the interpolation of global mean temperature anomaly data, a key step in reconstructing continuous climate trends from discrete observations. Interpolation techniques are essential for filling gaps in data, smoothing noise, and enabling further statistical analysis \cite{atkinson1989introduction, burden2011numerical}. We compare several classical numerical interpolation methods, including local Newton forward and backward interpolation, local Lagrange interpolation, and local polynomial regression. Each method is evaluated in terms of accuracy and suitability for climate data analysis.

A distinctive aspect of our approach is the use of local, pointwise interpolation: for each prediction year, the interpolation is performed using a window of the nearest neighbors, rather than fitting a global model. This local strategy is designed to increase accuracy, reduce the risk of overfitting or oscillations (such as Runge's phenomenon), and adapt to local variations in the data. We also systematically compare the methods over different training intervals, such as [1950, 2020] and [1960, 2010], to assess their stability and generalization across time.

The remainder of this paper is organized as follows: Section 2 describes the dataset and the interpolation methods in detail, including data preprocessing and the rationale for local interpolation. Section 3 presents the results of applying these methods to real-world climate data. Section 4 discusses the comparative performance of the methods, and Section 5 concludes with key findings and recommendations.
\section{Methods}

\subsection{Dataset Description and Preprocessing}

We use the ERA5-Land reanalysis dataset \cite{ERA5} that provides hourly 2-meter air temperature data over Iran. For this study, we extract the temperature values at 10 AM local time for the summer months (June, July, August) for each year in the period 1950--2020. The data is spatially gridded, covering the region of Iran with a regular latitude-longitude grid.

Prior to analysis, we perform data cleaning by removing any missing values (NaNs) from the temperature arrays. For each grid point, we compute the mean summer temperature at 10 AM for each year. This results in a time series of summer daytime temperatures for each location. To facilitate fair comparison between interpolation methods, we generate uniformly spaced sample points using linear interpolation, ensuring that all methods operate on the same input data.

\subsection{Choice of Training Intervals}

To evaluate the robustness and generalization of each interpolation method, we conduct experiments on two different training intervals: [1950, 2020] and [1960, 2010]. The first interval covers the full range of recent climate data, including boundary years where interpolation is typically more challenging. The second interval focuses on the central portion of the data, allowing us to assess method performance away from the boundaries and to compare how well each method generalizes to unseen years.

\subsection{Local, Pointwise Interpolation Strategy}

For each target year and location, interpolation is performed using a local window of the 8 nearest neighbors in time. This local approach offers several advantages:
\begin{itemize}
    \item \textbf{Increased accuracy:} By focusing on nearby data, the interpolant better captures local trends and reduces the influence of distant, potentially irrelevant points.
    \item \textbf{Reduced oscillations:} Local polynomials are less prone to Runge's phenomenon and numerical instability than high-degree global polynomials.
    \item \textbf{Adaptability:} The method can flexibly adapt to local changes in the data, which is especially important for real-world, noisy climate records.
\end{itemize}
For each method, if insufficient neighbors are available, the window size or polynomial degree is automatically reduced to ensure numerical stability.

\subsection{Interpolation Techniques}

We implement and compare the following interpolation methods:
\begin{itemize}
    \item \textbf{Local Newton Forward Interpolation:} Constructs an interpolating polynomial using forward differences, suitable for points near the start of the interval \cite{atkinson1989introduction, burden2011numerical, garcia2022newton}.
    \item \textbf{Local Newton Backward Interpolation:} Uses backward differences, providing better accuracy near the end of the interval \cite{garcia2022newton}.
    \item \textbf{Local Lagrange Interpolation:} Builds a polynomial passing through the local window, offering flexibility and high local accuracy \cite{atkinson1989introduction, smith2020numerical, lee2019comparison}.
    \item \textbf{Local Polynomial Regression:} Fits a polynomial of specified degree to the local window using least squares, with normalization for numerical stability \cite{brown2021polynomial, press2007numerical}.
\end{itemize}

All methods are implemented in Python using \texttt{numpy}, \texttt{scipy}, and \texttt{matplotlib}. If polynomial fitting fails, the code falls back to linear interpolation to ensure robustness.

\subsection{Evaluation and Output}

For each method and interval, we predict summer 10 AM temperatures for each year in the evaluation range and compare them to the actual observed values. Results are visualized as plots and saved as CSV files for further analysis. Performance is quantified using root mean square error (RMSE).

To further reduce the impact of outliers and unrealistic values, all final temperature estimates were limited to the range $[-20, 100]$ degrees Celsius.
\section{Results}

To comprehensively evaluate the interpolation methods, we performed experiments using two different training intervals: [1950, 2020] and [1960, 2010]. The first interval covers the entire period of interest, while the second focuses on a central subset. This dual-interval approach allows us to assess both interpolation (within the training range) and extrapolation (outside the training range) performance. By comparing results from these intervals, we can better understand the stability, generalization, and boundary sensitivity of each method.

Figures~\ref{fig:forward}-\ref{fig:regression} show the interpolated global mean temperature anomalies from 1950 to 2020 using each method, trained on the full [1950, 2020] interval. Figures~\ref{fig:forward2}-\ref{fig:regression2} present the results when the models are trained only on [1960, 2010], thus requiring extrapolation for years outside this range.

\begin{figure}[htbp]
    \centering
    \includegraphics[width=0.8\textwidth]{../figs/Local_Newton_Forward_vs_actual[1950, 2020, 1].png}
    \caption{Local Newton Forward Interpolation vs Actual Data (1950--2020), trained on [1950, 2020]}
    \label{fig:forward}
\end{figure}

\begin{figure}[htbp]
    \centering
    \includegraphics[width=0.8\textwidth]{../figs/Local_Newton_Backward_vs_actual[1950, 2020, 1].png}
    \caption{Local Newton Backward Interpolation vs Actual Data (1950--2020), trained on [1950, 2020]}
    \label{fig:backward}
\end{figure}

\begin{figure}[htbp]
    \centering
    \includegraphics[width=0.8\textwidth]{../figs/Local_Lagrange_vs_actual[1950, 2020, 1].png}
    \caption{Local Lagrange Interpolation vs Actual Data (1950--2020), trained on [1950, 2020]}
    \label{fig:lagrange}
\end{figure}

\begin{figure}[htbp]
    \centering
    \includegraphics[width=0.8\textwidth]{../figs/Local_Polynomial_Regression_vs_actual[1950, 2020, 1].png}
    \caption{Local Polynomial Regression vs Actual Data (1950--2020), trained on [1950, 2020]}
    \label{fig:regression}
\end{figure}

\begin{figure}[htbp]
    \centering
    \includegraphics[width=0.8\textwidth]{../figs/Local_Newton_Forward_vs_actual[1960, 2010, 1].png}
    \caption{Local Newton Forward Interpolation vs Actual Data (1950--2020), trained on [1960, 2010]}
    \label{fig:forward2}
\end{figure}

\begin{figure}[htbp]
    \centering
    \includegraphics[width=0.8\textwidth]{../figs/Local_Newton_Backward_vs_actual[1960, 2010, 1].png}
    \caption{Local Newton Backward Interpolation vs Actual Data (1950--2020), trained on [1960, 2010]}
    \label{fig:backward2}
\end{figure}

\begin{figure}[htbp]
    \centering
    \includegraphics[width=0.8\textwidth]{../figs/Local_Lagrange_vs_actual[1960, 2010, 1].png}
    \caption{Local Lagrange Interpolation vs Actual Data (1950--2020), trained on [1960, 2010]}
    \label{fig:lagrange2}
\end{figure}

\begin{figure}[htbp]
    \centering
    \includegraphics[width=0.8\textwidth]{../figs/Local_Polynomial_Regression_vs_actual[1960, 2010, 1].png}
    \caption{Local Polynomial Regression vs Actual Data (1950--2020), trained on [1960, 2010]}
    \label{fig:regression2}
\end{figure}

Table~\ref{tab:rmse} summarizes the RMSE for each method and interval.

\begin{table}[htbp]
    \centering
    \begin{tabular}{lcc}
        \hline
        Method & RMSE (1950--2020) & RMSE (1960--2010) \\
        \hline
        Local Newton Forward & [VALUE1] & [VALUE2] \\
        Local Newton Backward & [VALUE3] & [VALUE4] \\
        Local Lagrange & [VALUE5] & [VALUE6] \\
        Local Polynomial Regression & [VALUE7] & [VALUE8] \\
        \hline
    \end{tabular}
    \caption{Root Mean Square Error (RMSE) for each interpolation method and interval.}
    \label{tab:rmse}
\end{table}
\section{Discussion}

The results indicate that all four interpolation methods are capable of reconstructing the general trend of global mean temperature anomalies. However, there are notable differences in their performance:

\begin{itemize}
    \item \textbf{Local Newton Forward and Backward:} These methods perform well near the boundaries of the data but may introduce oscillations or inaccuracies in the middle of the interval, especially when the underlying function is not well-approximated by low-degree polynomials \cite{atkinson1989introduction}.
    \item \textbf{Local Lagrange:} This method provides high accuracy in regions with dense data but can be sensitive to noise and may suffer from Runge's phenomenon if the window size is too large.
    \item \textbf{Local Polynomial Regression:} This approach offers a good balance between flexibility and robustness, effectively smoothing noise while capturing the underlying trend. It generally achieves the lowest RMSE among the methods tested, consistent with findings in the literature \cite{brown2021polynomial}.
\end{itemize}

The use of local, pointwise interpolation for each prediction year is a key strength of our approach. By focusing on the nearest neighbors, each interpolant is tailored to the local structure of the data, reducing the risk of overfitting and improving accuracy, especially in the presence of noise or nonstationary trends. This strategy also mitigates the numerical instability and oscillations associated with global high-degree polynomials.

Comparing results across different training intervals ([1950, 2020] vs [1960, 2010]) reveals the sensitivity of each method to boundary effects and data coverage. Methods that perform well in the central interval may degrade near the boundaries, highlighting the importance of interval selection in climate data analysis.

Additional practical steps, such as data cleaning, normalization within local windows, and robust fallback to linear interpolation, further enhance the reliability and interpretability of the results.
\section{Conclusion}

In this study, we compared several classical numerical interpolation techniques for reconstructing 2-meter air temperature at 10 AM during summer months in Iran. Our results demonstrate that while all methods can approximate the overall trend, local polynomial regression provides superior accuracy and robustness for regional temperature analysis \cite{brown2021polynomial, smith2020numerical, atkinson1989introduction}. The use of local, pointwise interpolation windows for each prediction year and location significantly improves accuracy and stability, especially in the presence of noise and nonstationary trends. The ERA5-Land reanalysis dataset \cite{ERA5} serves as a reliable benchmark for such studies.

By evaluating methods across different training intervals, we highlighted the importance of interval selection and the sensitivity of interpolation methods to boundary effects \cite{lee2019comparison, burden2011numerical}. Additional steps such as data cleaning, normalization, and robust error handling further contribute to the reliability of the analysis.

Future work may explore the application of advanced machine learning-based interpolation methods and the extension of these techniques to other climate variables and spatially resolved datasets.

\bibliographystyle{plain}
\bibliography{references}

\end{document}
