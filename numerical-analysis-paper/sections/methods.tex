\section{Methods}

\subsection{Dataset Description and Preprocessing}

We use the ERA5-Land reanalysis dataset \cite{ERA5} that provides hourly 2-meter air temperature data over Iran. For this study, we extract the temperature values at 10 AM local time for the summer months (June, July, August) for each year in the period 1950--2020. The data is spatially gridded, covering the region of Iran with a regular latitude-longitude grid.

Prior to analysis, we perform data cleaning by removing any missing values (NaNs) from the temperature arrays. For each grid point, we compute the mean summer temperature at 10 AM for each year. This results in a time series of summer daytime temperatures for each location. To facilitate fair comparison between interpolation methods, we generate uniformly spaced sample points using linear interpolation, ensuring that all methods operate on the same input data.

\subsection{Choice of Training Intervals}

To evaluate the robustness and generalization of each interpolation method, we conduct experiments on two different training intervals: [1950, 2020] and [1960, 2010]. The first interval covers the full range of recent climate data, including boundary years where interpolation is typically more challenging. The second interval focuses on the central portion of the data, allowing us to assess method performance away from the boundaries and to compare how well each method generalizes to unseen years.

\subsection{Local, Pointwise Interpolation Strategy}

For each target year and location, interpolation is performed using a local window of the 8 nearest neighbors in time. This local approach offers several advantages:
\begin{itemize}
    \item \textbf{Increased accuracy:} By focusing on nearby data, the interpolant better captures local trends and reduces the influence of distant, potentially irrelevant points.
    \item \textbf{Reduced oscillations:} Local polynomials are less prone to Runge's phenomenon and numerical instability than high-degree global polynomials.
    \item \textbf{Adaptability:} The method can flexibly adapt to local changes in the data, which is especially important for real-world, noisy climate records.
\end{itemize}
For each method, if insufficient neighbors are available, the window size or polynomial degree is automatically reduced to ensure numerical stability.

\subsection{Interpolation Techniques}

We implement and compare the following interpolation methods:
\begin{itemize}
    \item \textbf{Local Newton Forward Interpolation:} Constructs an interpolating polynomial using forward differences, suitable for points near the start of the interval \cite{atkinson1989introduction, burden2011numerical, garcia2022newton}.
    \item \textbf{Local Newton Backward Interpolation:} Uses backward differences, providing better accuracy near the end of the interval \cite{garcia2022newton}.
    \item \textbf{Local Lagrange Interpolation:} Builds a polynomial passing through the local window, offering flexibility and high local accuracy \cite{atkinson1989introduction, smith2020numerical, lee2019comparison}.
    \item \textbf{Local Polynomial Regression:} Fits a polynomial of specified degree to the local window using least squares, with normalization for numerical stability \cite{brown2021polynomial, press2007numerical}.
\end{itemize}

All methods are implemented in Python using \texttt{numpy}, \texttt{scipy}, and \texttt{matplotlib}. If polynomial fitting fails, the code falls back to linear interpolation to ensure robustness.

\subsection{Evaluation and Output}

For each method and interval, we predict summer 10 AM temperatures for each year in the evaluation range and compare them to the actual observed values. Results are visualized as plots and saved as CSV files for further analysis. Performance is quantified using root mean square error (RMSE).

To further reduce the impact of outliers and unrealistic values, all final temperature estimates were limited to the range $[-20, 100]$ degrees Celsius.