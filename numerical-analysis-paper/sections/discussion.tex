\section{Discussion}

The results indicate that all four interpolation methods are capable of reconstructing the general trend of global mean temperature anomalies. However, there are notable differences in their performance:

\begin{itemize}
    \item \textbf{Local Newton Forward and Backward:} These methods perform well near the boundaries of the data but may introduce oscillations or inaccuracies in the middle of the interval, especially when the underlying function is not well-approximated by low-degree polynomials \cite{atkinson1989introduction}.
    \item \textbf{Local Lagrange:} This method provides high accuracy in regions with dense data but can be sensitive to noise and may suffer from Runge's phenomenon if the window size is too large.
    \item \textbf{Local Polynomial Regression:} This approach offers a good balance between flexibility and robustness, effectively smoothing noise while capturing the underlying trend. It generally achieves the lowest RMSE among the methods tested, consistent with findings in the literature \cite{brown2021polynomial}.
\end{itemize}

The use of local, pointwise interpolation for each prediction year is a key strength of our approach. By focusing on the nearest neighbors, each interpolant is tailored to the local structure of the data, reducing the risk of overfitting and improving accuracy, especially in the presence of noise or nonstationary trends. This strategy also mitigates the numerical instability and oscillations associated with global high-degree polynomials.

Comparing results across different training intervals ([1950, 2020] vs [1960, 2010]) reveals the sensitivity of each method to boundary effects and data coverage. Methods that perform well in the central interval may degrade near the boundaries, highlighting the importance of interval selection in climate data analysis.

Additional practical steps, such as data cleaning, normalization within local windows, and robust fallback to linear interpolation, further enhance the reliability and interpretability of the results.