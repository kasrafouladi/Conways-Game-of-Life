\section{Introduction}

Accurate analysis of near-surface air temperature is crucial for understanding regional climate patterns and their impacts, especially in countries with diverse climates such as Iran \cite{hansen2010global}. In this study, we focus on the 2-meter air temperature at 10 AM during the summer months (June, July, August) across Iran. This specific metric is important for assessing heat exposure and its effects on human health, agriculture, and energy demand.

Our data source is the ERA5-Land reanalysis dataset \cite{ERA5}, which provides gridded, hourly temperature values for the region of interest. We extract the relevant data for the summer months and the 10 AM time step for each year in the study period.

The main goal of this project is to interpolate and analyze the spatial and temporal patterns of summer daytime temperatures in Iran. We compare several classical numerical interpolation methods, including local Newton forward and backward interpolation, local Lagrange interpolation, and local polynomial regression \cite{atkinson1989introduction, burden2011numerical, smith2020numerical, johnson2018introduction, lee2019comparison, brown2021polynomial, garcia2022newton, press2007numerical}. Each method is evaluated in terms of accuracy and suitability for regional climate data analysis.

A distinctive aspect of our approach is the use of local, pointwise interpolation: for each prediction year and location, the interpolation is performed using a window of the nearest neighbors, rather than fitting a global model. This local strategy is designed to increase accuracy, reduce the risk of overfitting or oscillations (such as Runge's phenomenon), and adapt to local variations in the data. We also systematically compare the methods over different training intervals, such as [1950, 2020] and [1960, 2010], to assess their stability and generalization across time.

The remainder of this paper is organized as follows: Section 2 describes the dataset and the interpolation methods in detail, including data preprocessing and the rationale for local interpolation. Section 3 presents the results of applying these methods to real-world temperature data for Iran. Section 4 discusses the comparative performance of the methods, and Section 5 concludes with key findings and recommendations.