\section{Introduction}

Accurate analysis of climate data is crucial for understanding global warming and its impacts. One of the most widely used metrics in climate science is the \textit{temperature anomaly}, which represents the deviation of the observed temperature from a reference value or baseline period \cite{gistemp, hansen2010global}. This approach allows researchers to compare temperature changes across different regions and time periods, independent of local climate variability.

In this project, we focus on the interpolation of global mean temperature anomaly data, a key step in reconstructing continuous climate trends from discrete observations. Interpolation techniques are essential for filling gaps in data, smoothing noise, and enabling further statistical analysis \cite{atkinson1989introduction, burden2011numerical}. We compare several classical numerical interpolation methods, including local Newton forward and backward interpolation, local Lagrange interpolation, and local polynomial regression. Each method is evaluated in terms of accuracy and suitability for climate data analysis.

A distinctive aspect of our approach is the use of local, pointwise interpolation: for each prediction year, the interpolation is performed using a window of the nearest neighbors, rather than fitting a global model. This local strategy is designed to increase accuracy, reduce the risk of overfitting or oscillations (such as Runge's phenomenon), and adapt to local variations in the data. We also systematically compare the methods over different training intervals, such as [1950, 2020] and [1960, 2010], to assess their stability and generalization across time.

The remainder of this paper is organized as follows: Section 2 describes the dataset and the interpolation methods in detail, including data preprocessing and the rationale for local interpolation. Section 3 presents the results of applying these methods to real-world climate data. Section 4 discusses the comparative performance of the methods, and Section 5 concludes with key findings and recommendations.